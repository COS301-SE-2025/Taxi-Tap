\documentclass[12pt]{article}
\usepackage[a4paper, margin=1in]{geometry}
\usepackage{enumitem}
\usepackage{titlesec}
\usepackage{graphicx}
\usepackage{hyperref}

\titleformat{\section}{\Large\bfseries}{\thesection}{1em}{}
\titleformat{\subsection}{\large\bfseries}{\thesubsection}{1em}{}

\title{Taxi Tap\\Architectural Requirements Document}
\author{}
\date{}

\begin{document}

\maketitle

\section{3.6 Architectural Requirements Document}

\subsection{3.6.1 Architectural Design Strategy}
\textbf{Strategy Chosen:} \textit{Decomposition via Feature-Driven Development (FDD)}

Taxi Tap is built using a modular, feature-based decomposition strategy. Each functional system (e.g., User System, Vehicle System, Trip System) is designed, tested, and deployed independently. This strategy allows for:

\begin{itemize}
  \item Clear modularity and separation of concerns
  \item Parallel development and testing per feature
  \item Easy onboarding and maintainability
  \item Reduced risk when scaling or introducing new features
\end{itemize}

This approach accelerates development while ensuring traceability, maintainability, and scalability.

\subsection{3.6.2 Architectural Strategies}
\textbf{Chosen Style:} \textit{Microkernel Architecture (Plug-in Style)}

Each feature system in Taxi Tap functions as a plug-in module extending a core backend. This microkernel architecture suits our use case because:

\begin{itemize}
  \item Features can evolve and scale independently
  \item Systems are decoupled but unified by shared infrastructure
  \item Testing and deployment can occur at the system level
  \item Perfect fit for the Convex + Expo stack
\end{itemize}

\subsection{3.6.3 Architectural Quality Requirements}
The quality attributes for Taxi Tap are prioritized and defined as follows:

\begin{enumerate}
  \item \textbf{Scalability:} System must handle at least \textbf{100 concurrent ride requests} with a backend response time of under \textbf{100ms}.
  \item \textbf{Security:} All backend functions must enforce \textbf{role-based access control} (driver/passenger/admin).
  \item \textbf{Testability:} Each feature system must achieve \textbf{90\%+ test coverage} across unit and integration tests.
  \item \textbf{Availability:} Maintain at least \textbf{99.5\% uptime} under expected usage, with graceful degradation.
  \item \textbf{Maintainability:} All features must follow the FDD folder structure and be independently swappable.
\end{enumerate}

\subsection{3.6.4 Architectural Design and Pattern}
\textbf{Overview:} Taxi Tap is structured using a feature-driven, microkernel-inspired architecture. The diagram below illustrates this architecture.

\vspace{1em}
% \includegraphics[width=\linewidth]{architecture-diagram.png}
\framebox[\linewidth][c]{\textit{[Architecture Diagram Placeholder]}}
\vspace{1em}

\textbf{Components:}

\begin{itemize}
  \item \textbf{Expo Frontend:} Mobile-first interface using React Native
  \item \textbf{Convex Backend:} Serverless backend with modular mutations and schema
  \item \textbf{Convex Database:} Strongly-typed database used by each module
  \item \textbf{Feature Modules:} Each with its own schema, adapter, hook, and UI screen
\end{itemize}

This design provides modularity, scalability, and testability with minimal DevOps complexity.

\subsection{3.6.5 Architectural Constraints}
\begin{itemize}
  \item \textbf{Client Constraints:} Must remain within the AWS Free Tier; performance must be maintained under low-cost infrastructure.
  \item \textbf{Deployment Constraints:} Fully serverless; no Docker/Kubernetes; must deploy via CI/CD with minimal setup.
  \item \textbf{Security Constraints:} Only verified users may access trip, payment, or GPS functionality.
  \item \textbf{Latency Constraints:} Real-time location updates must occur under \textbf{1 second}.
  \item \textbf{Scalability Constraints:} Design must accommodate scaling to 1,000+ users without architectural changes.
\end{itemize}

\subsection{3.6.6 Technology Choices}

\subsubsection*{Backend Platform}
\begin{tabular}{|l|p{6cm}|p{6cm}|}
\hline
\textbf{Option} & \textbf{Pros} & \textbf{Cons} \\
\hline
Convex & Fully serverless, fast dev, native React support & New ecosystem, TypeScript only \\
\hline
Firebase & Realtime syncing, easy integration & Poor test tooling, security rule complexity \\
\hline
AWS Lambda & Highly scalable, mature & Complex CI/CD, requires DevOps setup \\
\hline
\textbf{Chosen:} Convex & \multicolumn{2}{l|}{Perfect fit for modular, testable architecture. Free tier-friendly.} \\
\hline
\end{tabular}

\vspace{1em}

\subsubsection*{Frontend Platform}
\begin{tabular}{|l|p{6cm}|p{6cm}|}
\hline
\textbf{Option} & \textbf{Pros} & \textbf{Cons} \\
\hline
Expo (React Native) & Fast prototyping, hot reload, cross-platform & Slightly heavier bundles \\
\hline
Flutter & Beautiful UI, good performance & Slower iteration, Dart-only \\
\hline
Native iOS/Android & Highest performance & High dev effort, no code sharing \\
\hline
\textbf{Chosen:} Expo & \multicolumn{2}{l|}{Fastest mobile-first path with TypeScript and Convex integration.} \\
\hline
\end{tabular}

\vspace{1em}

\subsubsection*{Database}
\begin{tabular}{|l|p{6cm}|p{6cm}|}
\hline
\textbf{Option} & \textbf{Pros} & \textbf{Cons} \\
\hline
Convex DB & Type-safe, built for Convex, no config & Smaller community \\
\hline
Firestore & Realtime, battle-tested & Complex security model \\
\hline
Supabase & Postgres-based, open source & Overhead for micro-systems \\
\hline
\textbf{Chosen:} Convex DB & \multicolumn{2}{l|}{Natively integrated with our serverless logic.} \\
\hline
\end{tabular}

\vspace{1em}

\subsubsection*{Payment Processor}
\begin{tabular}{|l|p{6cm}|p{6cm}|}
\hline
\textbf{Option} & \textbf{Pros} & \textbf{Cons} \\
\hline
Yoco & Local SA support, fast onboarding, easy to integrate & Limited advanced payment flows \\
\hline
Paystack & Clean APIs, good reliability & Limited card support in SA \\
\hline
Stripe & Powerful API, subscriptions & International fees, SA limitations \\
\hline
\textbf{Chosen:} Yoco & \multicolumn{2}{l|}{Best fit for local payments in South Africa. Simple, effective, mobile-friendly.} \\
\hline
\end{tabular}

\end{document}
